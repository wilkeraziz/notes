\section{Bernoulli}

Suppose $X$ takes on values in the set $\{0, 1\}$, then we say $X$ is Bernoulli-distributed 
\begin{equation}\label{eq:bern-pmf}
X \sim \Bern(\theta)
\end{equation}
where $0 < \theta < 1$ the Bernoulli parameter.
The Bernoulli pmf is
\begin{equation}
p(x; \theta) = \Bern(X=x|\theta) = \theta^x (1-\theta)^{(1-x)}
\end{equation}
and therefore the Bernoulli parameter corresponds to the probability of the positive class---i.e. $P_X(X=1)=\theta$.

We now derive the maximum likelihood estimate of the parameter $\theta$.
We start by rewriting the objective (\ref{eq:mle}) in terms for the Bernoulli  pmf (\ref{eq:bern-pmf})
\begin{subequations}
\begin{align}
\theta^\star &= \argmax_{\theta \in [0, 1]} ~ \mathcal L(\theta)  \\
 &= \argmax_{\theta \in (0, 1)} ~ \sum_{i=1}^n \log p(x_i; \theta) \label{bern:pmf-in}\\
 &= \argmax_{\theta \in (0, 1)} ~ \sum_{i=1}^n x_i \log \theta + (1-x_i)\log (1-\theta) \label{bern:log-property} \\
 &= \argmax_{\theta \in (0, 1)} ~ \log \theta \underbrace{\left(\sum_{i=1}^n x_i \right)}_{n_1} + \log (1-\theta) \underbrace{\left(\sum_{i=1}^n 1 - x_i\right)}_{n_0} \\
 &= \argmax_{\theta \in (0, 1)} ~ n_1 \log \theta + n_0 \log (1-\theta)
\end{align}
\end{subequations}
where we use $n_1$ for the number of positive observations and $n_0$ for the number of negative observations---and note that $n=n_1 + n_0$ is the total number of observations.

Now we find the first derivative of $\mathcal L(\theta)$ with respect to $\theta$:
\begin{subequations}
\begin{align}
	\dv{\mathcal L(\theta)}{\theta} &= \dv{\theta} \left[ n_1 \log \theta + n_0 \log (1-\theta) \right] \\
	 &= n_1 \dv{\theta} \log \theta + n_0 \dv{\theta} \log (1-\theta) \\
	 &= n_1 \dv{\theta} \log \theta + n_0 \underbrace{\dv{(1-\theta)} \log (1-\theta) \dv{(1-\theta)}{\theta}}_{\text{chain rule of derivatives}}\\
	 &= \frac{n_1}{\theta} + \frac{n_0}{1-\theta}(-1) 
\end{align}
\end{subequations}
Setting the derivative to $0$ and solving for $\theta$ gives us the MLE solution.
\begin{subequations}
\begin{align}
	0 &= \frac{n_1}{\theta} + \frac{n_0}{1-\theta}(-1)\\	
	 &= \frac{n_1 (1 - \theta) - n_0 \theta}{\theta(1-\theta)} \label{eq:bern-no-zero}\\
	 &= n_1 (1 - \theta) - n_0 \theta\\
	 &= n_1 - n_1 \theta - n_0 \theta\\
	n_1 &= \theta(n_1 + n_0)\\
	\theta &= \frac{n_1}{n_1 + n_0}  \\
	&= \frac{n_1}{n}
\end{align}
\end{subequations}
Note that in (\ref{eq:bern-no-zero}) we have to assume that the denominator $\theta(1-\theta) \neq 0$, which excludes $\theta=0$ and $\theta=1$. That explains why we define the Bernoulli parameter in the open interval $(0, 1)$. But note that this is not a bad thing, as $\theta=0$ or $\theta=1$ would lead to $X$ being deterministic. 