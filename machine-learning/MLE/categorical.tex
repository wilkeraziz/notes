\section{Categorical}

Suppose $X$ takes on values in the discrete interval $[1, k]$, then we say $X$ is Categorically-distributed 
\begin{equation}\label{eq:cat-pmf}
X \sim \Cat(\theta_1, \ldots, \theta_k)
\end{equation}
where $\theta_1, \ldots, \theta_k$ are the Categorical parameters subject to $\theta_x > 0$ and $\sum_{x=1}^k \theta_x = 1$. 
The Categorical pmf is
\begin{equation}
p(a; \theta_1^k) = \Cat(X=a|\theta_1, \ldots, \theta_k) = \prod_{x=1}^k \theta_x^{\delta_{xa}}
\end{equation}
where $\delta{ij}$ is the \href{https://en.wikipedia.org/wiki/Kronecker_delta}{Kronecker delta} and therefore each Categorical parameter corresponds to the probability of the respective category---i.e. $P_X(X=x) = \theta_x$.

We now derive the maximum likelihood estimate of the parameters $\theta_1^k$.
We start by rewriting the objective (\ref{eq:mle}) in terms for the Categorical  pmf (\ref{eq:bern-pmf})
\begin{subequations}
\begin{align}
\theta^\star &= \argmax_{\theta_1^k \in \Delta} ~ \mathcal L(\theta_1^k)  \\
 &= \argmax_{\theta_1^k\in \Delta} ~ \sum_{i=1}^n \log p(x_i; \theta_1^k) \label{bern:pmf-in}\\
 &= \argmax_{\theta_1^k\in \Delta} ~ \sum_{i=1}^n  \log \prod_{x=1}^k \theta_x^{\delta_{x x_i}} \\
% &= \argmax_{\theta_1^k\in \Delta} ~ \sum_{i=1}^n  \sum_{x=1}^k \log \theta_x^{\delta_{x x_i}}\\
 &= \argmax_{\theta_1^k\in \Delta} ~ \sum_{i=1}^n \sum_{x=1}^k \delta_{x x_i} \log \theta_x\\
% &= \argmax_{\theta_1^k\in \Delta} ~ \sum_{x=1}^k \sum_{i=1}^n \delta_{x x_i} \log \theta_x\\
 &= \argmax_{\theta_1^k\in \Delta} ~ \sum_{x=1}^k \log \theta_x \underbrace{\sum_{i=1}^n \delta_{x x_i}}_{n_x} \\
 &= \argmax_{\theta_1^k\in \Delta} ~ \sum_{x=1}^k n_x \log \theta_x 
\end{align}
\end{subequations}
where we denote the number of observations of class $x \in [1, k]$ by $n_x$.

To avoid optimising the constrained objective, where $\sum_{x=1}^k \theta_x = 1$, we employ a \href{https://en.wikipedia.org/wiki/Lagrange_multiplier}{Lagrange multiplier} $\lambda$ such that the new objective is
\begin{subequations}
\begin{align}
\theta^\star &= \argmax_{\theta_1^k \in \mathbb R^k} ~ \underbrace{\mathcal L(\theta_1^k) - \lambda \left[ \left(\sum_{x=1}^k \theta_x\right) - 1\right]}_{\mathcal L(\theta_1^k, \lambda)} \\
 &~\text{s.t. } \theta_x > 0 \text{ for }x \in [1, k] \nonumber
\end{align}
\end{subequations}


Now we find the first partial derivative of  $L(\theta_1^k, \lambda)$ with respect to $\lambda$
\begin{subequations}
\begin{align}
	\pdv{\lambda}\mathcal L(\theta_1^k, \lambda) &= \pdv{\lambda} \left\{ \sum_{x=1}^k n_x \log \theta_x  - \lambda \left[ \left(\sum_{x=1}^k \theta_x\right) - 1 \right] \right\} \\
	&= \cancelto{0}{\pdv{\lambda} \left\{ \sum_{x=1}^k n_x \log \theta_x  \right\} } - \left[ \left(\sum_{x=1}^k \theta_x\right) - 1 \right] \\
	&= 1 - \sum_{x=1}^k \theta_x
\end{align}
\end{subequations}
where setting the derivative to zero yields
\begin{equation} \label{eq:cat-const}
	\sum_{x=1}^k \theta_x = 1 \quad . 
\end{equation}

We now turn to the first partial derivative of $L(\theta_1^k, \lambda)$ with respect to $\theta_j$ for $j \in [1, k]$
\begin{subequations}
\begin{align}
	\pdv{\theta_j}\mathcal L(\theta_1^k, \lambda) &= \pdv{\theta_j} \left\{ \sum_{x=1}^k n_x \log \theta_x  - \lambda \left[ \left(\sum_{x=1}^k \theta_x\right) - 1 \right] \right\} \\
	&= \sum_{x=1}^k n_x  \pdv{\theta_j} \log \theta_x - \lambda  \sum_{x=1}^k \pdv{\theta_j} \theta_x \\
	&= \sum_{x=1}^k n_x \frac{\delta{jx}}{\theta_x} - \lambda  \sum_{x=1}^k \delta_{jx} \\
	&= \frac{n_j}{\theta_j} - \lambda
\end{align}
\end{subequations}
where setting the derivative to zero yields
\begin{subequations}
\begin{align}
0 &= \frac{n_j}{\theta_j} -\lambda \\
\lambda &= \frac{n_j}{\theta_j} \label{eq:cat-not-zero} \\ 
\theta_j &= \frac{n_j}{\lambda} \label{eq:thetaj}
\end{align}
\end{subequations}

Now substituting (\ref{eq:thetaj}) into (\ref{eq:cat-const}) we have
\begin{subequations}
\begin{align}
\sum_{x=1}^k \theta_x  = \sum_{x=1}^k \frac{n_x}{\lambda} = \frac{1}{\lambda} \sum_{x=1}^k n_x = \frac{1}{\lambda} n  = 1
\end{align}
\end{subequations}
and therefore $\lambda = \frac{1}{n}$. And finally, substituting $\lambda$ in (\ref{eq:thetaj})
\begin{equation}
	\theta_j =  \frac{n_j}{n} \label{eq:cat-solution}
\end{equation}
yields the maximum likelihood estimate.
Note that in (\ref{eq:cat-not-zero}) we have to assume $\theta_j$ is not 0, and that's why we defined the categorical pmf for $\theta_j \in \mathbb R_{>0}$. Also note that the solution in (\ref{eq:cat-solution}) is strictly positive as long as $n_j > 0$.


